\subsection*{Resumo}

O projeto se trata de um trabalho prático do curso Programação e Desenvolvimento de Software 2 da Universidade Federal de Minas Gerais. O tema escolhido foi um Twitter que foi desenvolvido em C++ com um banco de dados My\+S\+QL.

\#\# Dependências 
\begin{DoxyCode}
> sudo apt install g++
> sudo apt install build-essential
> sudo apt install mysql
> sudo apt install libmysqlcppconn-dev
> sudo apt install libmysqlclient15-dev //necessário em certas distros
\end{DoxyCode}
 \begin{quote}
O conector My\+S\+Q\+L/\+C++ também pode ser encontrado no link\+: \href{https://dev.mysql.com/downloads/connector/cpp/}{\tt https\+://dev.\+mysql.\+com/downloads/connector/cpp/} \end{quote}


Configure na {\itshape \hyperlink{_abstract_connection_8cpp}{Abstract\+Connection.\+cpp}} o endereço do seu banco de dados, usuário, senha e schema (este deve ser criado com o script S\+QL na raiz do projeto)

\subsection*{Execução}

Na pasta {\itshape 2018-\/2-\/grupo19/cmake-\/build-\/debug} encontra-\/se os arquivos {\itshape Makefile} e o executável {\itshape 2018-\/2-\/grupo19}. Para compilar novamente o projeto, basta digitar\+: {\itshape make}. Para executar\+: $\ast$./2018-\/2-\/grupo19$\ast$. {\bfseries O projeto deve ser utilizado em um ambiente Linux compatível com as dependências citadas acima (Ubuntu, por exemplo)}

Recomenda-\/se o uso da I\+DE C\+Lion, da Jetbrains para fazer qualquer modificaço no projeto, já que o mesmo foi integralmente desenvolvido nesse ambiente. \begin{quote}
Link para download do C\+Lion\+: \href{https://www.jetbrains.com/clion/download/}{\tt https\+://www.\+jetbrains.\+com/clion/download/} \end{quote}


\subsection*{Relatório}

No nosso C\+RC, pensamos numa classe central que seria o Feed, responsável por fazer a função de interface com o usuário e a correlação das outras classes, como Usuário e Tweet, onde o Feed demandaria as funções dessas classes, por exemplo e assim elas iriam se complementando e trabalhando juntas. As classes “\+Menção”, que são as Userstags na modelagem, Comentários e as Hashtags seriam classes menores apenas para complementar as funções do Tweet como conhecemos, sendo criadas para fazer parte de uma classe, como uma de suas variáveis ou como um de seus vetores principais.

Identificamos que, para trabalharmos com uma grande quantidade de usuários, tweets, comentários, curtidas, que demandavam uma grande quantidade de dados para serem armazenados e uma necessidade de organização e pesquisa mais facilitada. Devido a isso, preferimos utilizar um Banco de Dados em vez de arquivos-\/texto ou binários, pois além desses fatores mencionados, a utilização simultânea de dados é possível e o “id” das variáveis eram gerados automaticamente. As funções do Banco de Dados também permitiam pesquisa, escrita e, principalmente, atualização de dados facilitada, nos permitindo ter mais tempo para implementar as demais utilidades do Twitter.

Banco de Dados selecionado foi o My\+S\+QL e na sua modelagem percebemos que uma tabela para todas as classes criadas no C\+RC era necessário, menos o Feed, que posteriormente tratamos apenas como uma das diversas interfaces (“telas”) criadas. Surgiu, também, a necessidade de criar uma tabela para Folowee, pois não tínhamos como ir adicionamos inúmeros elementos na tabela de usuários para representar quem ele segue, cada linha nova é utilizada para representar uma nova conta e adicionar inúmeras colunas era inviável. Além disso, os Likes do Tweet e os Likes do Comentário demandaram uma nova tabela pelo mesmo motivo de Folowee, são dados que são referentes a uma classe, mas que não tem número definido de ocorrências, logo era necessário armazenar o id do usuário que curtiu o comentário ou o tweet em num lugar diferente.

Para a implementação do sistema propriamente dito, começamos com as classes e seus getters e setters, que são funções básicas mas necessárias. Depois disso, começamos a seguir nossa Users Stories para ter uma base de como começar a implementar e no que focar primeiro, seguindo também uma lógica do próprio usuário, que começa na tela de Login, depois vai pro seu feed, onde tem acesso aos tweets de quem segue e pode selecionar uma ação específica como tweetar, ver seu perfil, entre outras. Com isso, decidimos que as interfaces seriam responsáveis por printar dados e receber comandos dos usuários, consequentemente, todas teriam que processar os comandos e exibir alguma coisa, surgindo então funções básicas para uma classe, que seria especializada em cada tela propriamente dita.

Cada tela por sua vez, ao processar a informação deveria, a partir do usuário que logado, demandar ações das classes definidas no C\+RC e na Modelagem do Banco de Dados, que seriam responsáveis por, a partir da opção selecionada, pesquisar, ou inserir um novo dado no Banco de Dados e retornar algo para ser mostrado nas interfaces. Com isso surgiu a ideia da classe Sessão, responsável por armazenar o usuário logado para servir de referência para as demais classes. A classe base Abstract\+Connection também surgiu a partir dessa logística, pois ela é responsável pela função e os ponteiros que abrem e fecham a conexão com o banco de dados, acabando com diversos códigos repetidos no sistema. 